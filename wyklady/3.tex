\documentclass[10pt]{beamer}
\usepackage[cache=false]{minted}
\usepackage[T1]{fontenc}
\usepackage[utf8]{inputenc}
\usepackage[english,polish]{babel}
\usepackage{polski}
\usepackage{graphicx}
\usepackage{amsthm}
\graphicspath{ {./images/} }
\usetheme[progressbar=frametitle]{metropolis}
\usepackage{appendixnumberbeamer}
\usepackage{booktabs}
\usepackage[scale=2]{ccicons}
\usepackage{fancyvrb}

\usepackage{pgfplots}
\usepackage{hyperref}
\hypersetup{
    colorlinks=true,
    linkcolor=blue,
    filecolor=magenta,      
    urlcolor=cyan,
}
\usepgfplotslibrary{dateplot}

\usepackage{xspace}
\newcommand{\themename}{\textbf{\textsc{metropolis}}\xspace}
\newcommand{\quotes}[1]{``#1''}
\title{Golang - wstęp}
\date{\today}
\author{Adam Pietrzak}
\institute{}

\begin{document}

\maketitle

\begin{frame}{Spis treści}
  \setbeamertemplate{section in toc}[sections numbered]
  \tableofcontents%[hideallsubsections]
\end{frame}

\section[Interface]{Interface}
\begin{frame}[fragile]
    \frametitle{Interface}
    Interface jest zbiorem definicji funkcji, a dokładniej ich wyglądu.

    Przykładowy wygląd interface:

    \begin{minted}[fontsize=\footnotesize]{Go}
        type MyInterface interface {
            MyMethod(v int) int64
        }
    \end{minted}

    \begin{minted}[fontsize=\footnotesize]{Go}
        type A struct {}

        func (a A) MyMethod (v int) int64 {
            return 2
        }

        var a MyInterface = A{}
        func b () {
            t := a.MyMethod(12)
        }
    \end{minted}
\end{frame}

\begin{frame}[fragile]
    \frametitle{Interface II}
    Interface jest spełniony wtedy, kiedy wszystkie metody występują w danym typie.
    Nie istnieje dodatkowy zapis mówiący, że dany typ implementuje interace.

    Interface pod spodem jest zapisany jako \quotes{typ} oraz \quotes{wartość}.
    Jest to bardzo ważne przy używanie interface.

    Wartość interface może być \quotes{nil}:
    \begin{minted}[fontsize=\footnotesize]{Go}
        var a MyInterface = nil
    \end{minted}
\end{frame}

\begin{frame}[fragile]
    \frametitle{Interface III}
    \quotes{nil} jako wartość interface jest ważna. Po pierwsze, można na niej uruchomić
    metodę!

    \begin{minted}[fontsize=\footnotesize]{Go}
        vat t *MyType
        var a MyInterface = t
        a.MyMethod()
    \end{minted}

    Nie jest to błąd w GoLang. W takiej sytuacji dostaniemy jako \quotes{receiver} nil:

    \begin{minted}[fontsize=\footnotesize]{Go}
        func (m *MyType) MyMethod() {
            if m == nil {

            }
            // ...
        }
    \end{minted}
\end{frame}

\begin{frame}[fragile]
    \frametitle{Interface IV}
    Wywołanie interface, który nie ma typu (jest \quotes{nil}), już pokaże błąd:

    \begin{minted}[fontsize=\footnotesize]{Go}
        var a MyInterface = nil
        a.MyMethod()
    \end{minted}

    Dodatkowo, trzymanie typu w interface powoduje problemy z porównaniami:
    \begin{minted}[fontsize=\footnotesize]{Go}
        type MyError struct {}

        func (m MyError) Error () string {
            return ""
        }
        func test () error {
            var e *MyError = nil
            return e
        }
        func b () {
            if test() == nil {
            }
        }
    \end{minted}
\end{frame}

\begin{frame}[fragile]
    \frametitle{Interface V}
    Istnieje w GoLangu typ: \quotes{any}. Jest nim zapis:
    \begin{minted}[fontsize=\footnotesize]{Go}
        var a interface{}
    \end{minted}

    Ponieważ wszystko implementuje pusty interface, wszystko pasuje do tego zapisu.

    Zapis ten powoduje problemy z typowaniem, ponieważ wszystko tutaj pasuje.

    Dodatkowo, możemy mapować interface na ich typ wewnętrzny:

    \begin{minted}[fontsize=\footnotesize]{Go}
        var a interface{} = 23
        c, ok := a.(int)
    \end{minted}

    \begin{alertblock}{Uwaga}
        Typy muszą się zgadzać 1:1.
    \end{alertblock}

    \begin{block}{ok}
        \quotes{ok} jest opcjonalne.
    \end{block}
\end{frame}

\begin{frame}[fragile]
    \frametitle{Interface VI}
    Istnieje dodatkowy zapis w \quotes{switch} ułatwiający mapowanie typów, gdy
    funkcja może przyjąć więcej niż jeden typ:

    \begin{minted}[fontsize=\footnotesize]{Go}
        var a interface{} = 23

        switch a.(type) {
            case int:
                // Code
            case int64:
                // Code
        }
    \end{minted}

    \begin{alertblock}{Uwaga!}
        Interface nie są dla tworzącego tylko dla używającego. Jeżeli nie ma 
        potrzeby tworzenia interface, nie należy udostępniać interface!
    \end{alertblock}
\end{frame}

\section[Własne typy]{Własne typy}
\begin{frame}[fragile]
    \frametitle{Własne typy}
    Golang pozwala tworzy własne typy. Daje to nam możliwość większej
    walidacji typów (typy muszą się zgadzać), oraz możliwości rozszerzenia typów:

    \begin{minted}[fontsize=\footnotesize]{Go}
        type A string
        func (a A) Test () {

        }

        type B map[string]string
    \end{minted}

    Własne typy możemy rozszerzać o metody.
    
    W łatwy sposób, możemy rzutować typy:

    \begin{minted}[fontsize=\footnotesize]{Go}
        var a A  = "A"
        var b string = string(a)
        var c A = A(b)
    \end{minted}
\end{frame}

\section[GoRoutines]{GoRoutines}
\begin{frame}[fragile]
    \frametitle{GoRoutines}
    \quotes{GoRoutines} jest sposobem na uruchamianie części kodu w innych wątkach.
    Dzięki temu, w łatwy sposób można wykonywać operacje równolegle.

    \begin{minted}[fontsize=\footnotesize]{Go}
        func a () {
            // Long running function
        }

        func b () {
            // Opetations
            go a()
            // operations
        }
    \end{minted}

    Funkcje te są uruchamiane w tej samej pamięci, więc należy synchronizować dostępy!
\end{frame}

\begin{frame}[fragile]
    \frametitle{Channel}
    \quotes{channel} jest sposobem na komunikację z \quotes{goroutine}. Dzięki temu, możemy synchronizować 
    wykonanie zadań, jak i również odbierać dane. 

    \quotes{channel} tworzymy za pomocą funkcji \quotes{make}. Drugi parametr robi z channela buffor.

    \begin{minted}[fontsize=\footnotesize]{Go}
    func sum(s []int, c chan int) {
        sum := 0
        for _, v := range s {
            sum += v
        }
        c <- sum // send sum to c
    }
    func main() {
        s := []int{7, 2, 8, -9, 4, 0}
        c := make(chan int)
        go sum(s[:len(s)/2], c)
        go sum(s[len(s)/2:], c)
        x, y := <-c, <-c // receive from c
        fmt.Println(x, y, x+y)
    }
    \end{minted}
\end{frame}

\begin{frame}[fragile]
    \frametitle{Channel II}
    \quotes{buffered channel} może zawierać w sobie więcej niż jedną wartość do odebrania.
    Ponieważ dopóki channel jest pełen, to wpisanie wartości blokuje wątek.

    \begin{minted}[fontsize=\footnotesize]{Go}
    func main() {
        ch := make(chan int, 2)
        ch <- 1
        ch <- 2
        fmt.Println(<-ch)
        fmt.Println(<-ch)
    }
    \end{minted}
\end{frame}

\begin{frame}[fragile]
    \frametitle{Channel III}
    Channele można używać z funkcją \quotes{range}, \quotes{len}, \quotes{cap}.
    Dodatkowo, można je zamykać:

    \begin{minted}[fontsize=\footnotesize]{Go}
    func fibonacci(n int, c chan int) {
        x, y := 0, 1
        for i := 0; i < n; i++ {
            c <- x
            x, y = y, x+y
        }
        close(c)
    }
    
    func main() {
        c := make(chan int, 10)
        go fibonacci(cap(c), c)
        for i := range c {
            fmt.Println(i)
        }
    }
    \end{minted}
\end{frame}

\begin{frame}[fragile]
    \frametitle{Channel IV}
    Istnieje jeszcze funkcja select, która wywołuje kod w zależności który channel
    jest do obsługi:

    \begin{minted}[fontsize=\footnotesize]{Go}
    func main() {
        tick := time.Tick(100 * time.Millisecond)
        boom := time.After(500 * time.Millisecond)
        for {
            select {
            case <-tick:
                fmt.Println("tick.")
            case <-boom:
                fmt.Println("BOOM!")
                return
            default:
                time.Sleep(50 * time.Millisecond)
            }
        }
    }
    \end{minted}

    \quotes{default} uruchamia się w sytuacji, gdy nie ma nic do obsługi.

    Gdy kilka channeli jest gotowych do odebrania, \quotes{select} wybierze losowo.
\end{frame}

\end{document}
