\documentclass[10pt]{beamer}
\usepackage[cache=false]{minted}
\usepackage[T1]{fontenc}
\usepackage[utf8]{inputenc}
\usepackage[english,polish]{babel}
\usepackage{polski}
\usepackage{graphicx}
\usepackage{amsthm}
\graphicspath{ {./images/} }
\usetheme[progressbar=frametitle]{metropolis}
\usepackage{appendixnumberbeamer}
\usepackage{booktabs}
\usepackage[scale=2]{ccicons}
\usepackage{fancyvrb}

\usepackage{pgfplots}
\usepackage{hyperref}
\hypersetup{
    colorlinks=true,
    linkcolor=blue,
    filecolor=magenta,      
    urlcolor=cyan,
}
\usepgfplotslibrary{dateplot}

\usepackage{xspace}
\newcommand{\themename}{\textbf{\textsc{metropolis}}\xspace}
\newcommand{\quotes}[1]{``#1''}
\title{Golang - wstęp}
\date{\today}
\author{Adam Pietrzak}
\institute{}

\begin{document}

\maketitle

\begin{frame}{Spis treści}
  \setbeamertemplate{section in toc}[sections numbered]
  \tableofcontents%[hideallsubsections]
\end{frame}

\section[GoLang]{GoLang}
\begin{frame}[fragile]
    \frametitle{Wstęp}
    GoLang (właściwie Go) - język programowania opracowany w 2009 na potrzeby Google.

    Głównymi elementami języka są:
    \begin{itemize}
        \item statyczne typowanie - wszystkie wartości mają określony typ
        \item silne typowanie - brak możliwości niejawnej zmiany typu
        \item kompilacja do jednego pliku wykonywalnego - zawsze powstaje jeden plik
        wykonywalny
    \end{itemize}
\end{frame}

\begin{frame}[fragile]
    \frametitle{Wstęp - maskotka}
    Maskotką i logiem języka Go jest "Go Gopher".

    \begin{center}
        \includegraphics[scale=5]{gopher.jpg}
    \end{center}
\end{frame}

\section[Składnia]{Składnia}
\begin{frame}[fragile]
    \frametitle{Składnia}
    Składnia Go jest podobna do języka C i Pythona (czasami).

    Każdy program składa się z funkcji main, lecz plik ją zawierający może być
    różnie nazwany.

    Program jest dzielony na paczki, gdzie każda paczka może zawierać więcej
    niż jeden plik źródłowy. Ważnym jest, że importując paczkę, importujemy wszystkie
    pliki. Dzieje się tak, dlatego że paczka jest kompilowana jako jedno.

    Dodatkowo, istnieją moduły. Moduły zawierają zbiór paczek ze sobą powiązanych,
    oraz są wdrażane razem. Każde repozytorium zawiera przynajmniej jeden moduł!

    \begin{minted}[fontsize=\footnotesize]{Go}
    package main

    import "fmt"

    func main() {
        fmt.Println("Hello, world.")
    }
    \end{minted}
\end{frame}

\begin{frame}[fragile]
    \frametitle{Składnia - pierwszy program}
    \begin{minted}[fontsize=\footnotesize]{Go}
    // Package information
    package main
    
    // Imports
    import "fmt"

    // Main function
    func main() {
        fmt.Println("Hello, world.")
    }
    \end{minted}

    \begin{block}{Nazewnictwo}
        W GoLangu nazywanie funkcji, zmiennych itp. zależy od ich użycia.
        Warto pamiętać, że jeżeli nazwa zawiera skrót, to nie mieszamy wielkich liter
        z małymi w tym skrócie: \quotes{xmlValue}, \quotes{XMLToSave}, \quotes{makeXMLHTTPValue},
        \quotes{xmlHTTPRequest}.

        To samo dotyczy \quotes{ID}.
    \end{block}
\end{frame}

\begin{frame}[fragile]
    \frametitle{Prywatność i publiczność}
    GoLang trzyma się jednej zasady, wszystko, co jest pisane z wielkiej litery,
    jest publiczne, wszystko co z małej litery, jest prywatne.

    Dotyczy to funkcji definiowanych na poziomie modułu, czy też
    pól w obiektach.

    Przykładowo, będąc w module \quotes{database}, zdefiniowałem funkcję:

    \begin{minted}[fontsize=\footnotesize]{Go}
        func add () {
            // implementation
        }
    \end{minted}

    W ramach modułu database, wszystkie pliki widzą i mogą używać tej funkcji.
    Jeżeli ktoś wywoła tą funkcję spoza modułu database, pojawi się błąd kompilacji.


    \begin{block}{Ciekawostka}
        Funkcja main jest pisana z małej litery.
    \end{block}
\end{frame}

\begin{frame}[fragile]
    \frametitle{Struktura plików}
    Go zmienia podejście do struktury plików. Pisząc kod w Javie, C++,
    JS zazwyczaj pojawia się katalog \quotes{src} zawierający kod aplikacji.

    Według GO, w katalogu głównym posiadamy plik go.mod, informujący co to jest
    za moduł, a w podkatalogach różne paczki. W ramach modułu paczki mogą się importować,
    dzięki czemu mamy do nich dostęp.

    \begin{alertblock}{Circular dependency}
        GoLang nie posiada możliwości rozwiązania circular dependency, więc
        należy uważać na importy.
    \end{alertblock}
\end{frame}

\begin{frame}[fragile]
    \frametitle{Go modules}
    Oryginalnie, GoLang nie pozwalał na tworzenie kodu poza zdefiniowaną zmienną
    środowiskową \quotes{GOPATH}. Czyli wszystkie programy, aplikacje, biblioteki
    musiały znajdować się w tej ścieżce.

    Na szczęście, od GO $1.14$ pojawiły się \quotes{Production ready} Go Modules,
    które pozwalają na tworzenie aplikacji w dowolnym miejscu na dysku.

    \begin{block}{$< 1.14$}
        Go Modules istniały już od 1.11, lecz zmieniały się w czasie.
    \end{block}
\end{frame}



\section[Język]{Podstawy języka}
\begin{frame}[fragile]
    \frametitle{Zmienne}
    Go posiada dwa sposoby definiowania zmiennych:

    Pierwszym z nich jest słówko kluczowe \quotes{var}.
    \begin{minted}[fontsize=\footnotesize]{Go}
        var i int = 0
    \end{minted}

    Dzięki takiemu zapisowi, możemy zdefiniować i zainicjalizować zmienną.
    Dużym plusem jest tutaj możliwość samej definicji, bez inicjalizacji:

    \begin{minted}[fontsize=\footnotesize]{Go}
        var i int
    \end{minted}

    Każda zmienna bez inicjalizacji posiada wartość domyślną (\quotes{zero value}):
    \begin{itemize}
        \item 0 - dla liczb
        \item \quotes{} - dla łańcucha znaków
        \item false - dla typów logicznych
    \end{itemize}

    Drugim plusem tej definicji jest jasny typ zmiennej.

    \begin{block}{Stałe}
        Aby zdefiniować stałą, używamy słowa \quotes{const}.
    \end{block}
\end{frame}

\begin{frame}[fragile]
    \frametitle{Zmienne}
    Drugim zapisem jest podstawienie \quotes{:=}. Dzięki niemu, nie musimy używać słowa
    kluczowego \quotes{var}.

    \begin{minted}[fontsize=\footnotesize]{Go}
        i := 0  // OK
    \end{minted}

    Podstawienie ma kilka wad:
    \begin{itemize}
        \item jeżeli zmienna istnieje, mamy błąd,
        \item brak jasnego typu,
        \item działa tylko w ciele funkcji
    \end{itemize}

    \begin{minted}[fontsize=\footnotesize]{Go}
        i := 0  // error

        func main () {
            i := 0 // ok

            i = 2 // ok

            i := 1 // error
            var test uint
            test = i // error, can not assign int -> uint
        }
    \end{minted}
\end{frame}

\begin{frame}[fragile]
    \frametitle{Typowanie i rzutowania}
    Go jest silnie typowane. Nie każdy język statyczny jest silnie typowany,
    przykładem jest C, gdzie poprzez wskaźnik można zmienić typ dowolnej zmiennej.

    Go jest inne, tutaj typ jest ostateczny. Nadanie jakiegoś typu zmiennej
    spowoduje, że żaden inny typ nie będzie to niej pasować.

    Przykładowo:
    \begin{minted}[fontsize=\footnotesize]{Go}
        var i int = 0
        var j int32 = 1
        var k int64 = 2
        var l string = ""

        i = l // error
        l = i // error
        j = k // error
        k = j // error
        i = j // error
    \end{minted}

    \begin{alertblock}{Własne typy}
        Dotyczy to także typów własnych!
    \end{alertblock}
\end{frame}

\begin{frame}[fragile]
    \frametitle{If}
    Ify są bardzo podobne do innych języków. Ich składnia nie wymaga
    nawiasów. Przykład:

    \begin{minted}[fontsize=\footnotesize]{Go}
        if a == 2 {
            return "OK"
        }
    \end{minted}

    Jednak, GoLang dodaje możliwość łączenia przypisania z warunkiem,
    co nam daje zapis:

    \begin{minted}[fontsize=\footnotesize]{Go}
        if a := 2; a == 2 {
            return "OK"
        }
        // ...
        if item, err := getItem(id); err == nil {
            return "OK"
        } else {
            log.Panic(err)
        }
    \end{minted}

    \begin{block}{Łączenie w praktyce}
        Zapis łączenia jest często stosowany w Go, dlatego należy go 
        rozumieć.
    \end{block}
\end{frame}

\begin{frame}[fragile]
    \frametitle{Pętla for}
    Pętla for mocno nie różni się od innych języków:

    \begin{minted}[fontsize=\footnotesize]{Go}
        for i := 0; i < 10; i++ {
            // Code
        }
        // Part of for can be skip and it goes to while
        for value < 1000 {
            value += getValue()
        }
    \end{minted}

    Dodatkowo, można iterować za pomocą słowa kluczowego \quotes{range}:
    \begin{minted}[fontsize=\footnotesize]{Go}
        for index, value := range myArray {
            // ...
        }
    \end{minted}
\end{frame}

\begin{frame}[fragile]
    \frametitle{Switch}
    Pomimo podobnego zapisu i z pozoru podobnego działania, switch w GoLangu
    daje dużo więcej możliwości:

    \begin{minted}[fontsize=\footnotesize]{Go}
        switch value {
            case 20:
                log.Print("20")
            case 40:
                log.Print("40")
            default:
                log.Print("Hmmm, not expected")
        }
    \end{minted}

    Do tego, można zrobić zapis:
    \begin{minted}[fontsize=\footnotesize]{Go}
        switch value := getValue(); value {
            // ...
        }
    \end{minted}
\end{frame}

\begin{frame}[fragile]
    \frametitle{Switch II}
    Oczywiście, możemy nawet pominąć warunek:
    \begin{minted}[fontsize=\footnotesize]{Go}
        value := 20 + rand.Int()
        switch {
        case value == 20:
            //
        case time.Now().Hour() < 10:
            //
        }
    \end{minted}

    Widać, że mamy dużą wolność wyboru w pisaniu switcha. Switch nie wymaga
    wartości stałych, czy też liczb do swojego działania. Dzięki temu
    daje łatwość użycia.

    Warto zauważyć, że nie używamy \quotes{break} na końcu \quotes{case}. Go
    dodaje je automatycznie.

    \begin{block}{Uwaga}
        Switch jest wykonywany od góry do dołu i kończy się przy pierwszym trafieniu.
        Aby wymusić kolejne warunki, należy użyć słowa kluczowego: \quotes{fallthrough}
        na końcu warunku.
    \end{block}
\end{frame}

\begin{frame}
    \frametitle{Enum}
    GoLang nie posiada implementacji Enuma. Jego zamiennik to zbiór stałych.
    Można przykład zobaczyć w ćwiczeniu 1.

    \begin{center}
        \includegraphics[scale=0.25]{sad_gophers.png}
    \end{center}
\end{frame}

\begin{frame}[fragile]
    \frametitle{Funkcje}
    Podstawowy zapis funkcji nie będzie odbiegać od innych języków:

    \begin{minted}[fontsize=\footnotesize]{Go}
        func MyFunction (arg1 int, arg2, arg3 string) bool {
            return true
        }
    \end{minted}

    Widzimy, że możemy skracać zapis wielu argumentów z tym samym typem.

    Dodatkowo, funkcja daje możliwość zwracania wielu danych:
    \begin{minted}[fontsize=\footnotesize]{Go}
    func MyFunction (arg1 int, arg2, arg3 string) (bool, int) {
        return true, 20
    }
    
    a, b := MyFunction(10, "a", "b")
    \end{minted}
    
    \begin{alertblock}{Nawias w zwracaniu danych}
        Nawias, który został dodany przy zwracanych danych jest wymagany
        dla więcej niż jednego wyjścia!
    \end{alertblock}
\end{frame}

\begin{frame}[fragile]
    \frametitle{Funkcje II}
    Dodatkowo, możemy tutaj zdefiniować nazwy zwracanych parametrów:

    \begin{minted}[fontsize=\footnotesize]{Go}
        func MyFunction (arg1 int, arg2, arg3 string) (test bool, value int) {
            if arg1 > 10 {
                test = true
                value = 20
            } else if arg1 == 10 {
                return true, 10
            }
            test = false
            return
        }
        
        a, b := MyFunction(10, "a", "b")
    \end{minted}

    \begin{alertblock}{Uwaga!}
        Funkcja która coś zwraca, zawsze musi wywołać słowo kluczowe \quotes{return}!
    \end{alertblock}

    \begin{block}{Kiedy nazywać, kiedy nie?}
        Nazwy parametrów powinno się używać w przypadku, gdy chcemy ułatwić
        zrozumienie co zwraca funkcja. Jeżeli zwracane wartości są jasne,
        nie ma sensu wykorzystywać tego sposobu.
    \end{block}
\end{frame}

\begin{frame}[fragile]
    \frametitle{Funkcje III}
    Często bywa tak, że chcemy reużyć nazwę zmiennej do wyjścia funkcji,
    która zwraca więcej niż jeden parametr:

    \begin{minted}[fontsize=\footnotesize]{Go}
        value, err := getValue()
        if err != nil {
            // handle error
        }

        savedValue, err := saveValue()
        _, err := saveValue() // Not it's error
    \end{minted}

    Jak widać Go pozwoli nam użyć skrócony zapis \quotes{:=} dopóki chociaż
    jedna zmienna będzie nowa!

    \begin{block}{Pomijanie parametrów wyjściowych}
        W GoLangu, nieużywane zmienne, importy powodują błędy kompilacji.
        Aby ich uniknąć, można użyć zapisu \quotes{\_ := myFunc()}
    \end{block}

    \begin{block}{Conistency}
        Wiele osób uważa, że takie wykluczenie psuje spójność języka.
    \end{block}
\end{frame}

\begin{frame}[fragile]
    \frametitle{Funkcje IV - errory}
    Go nie posiada wyjątków, przez co funkcje zwracają błędy.
    Błędy implementują interface \quotes{error}. Do stworzenia nowego błędu
    można użyć zapisu: \quotes{errors.New(string)}.

    Konwencja mówi, aby zwracać błędy jako ostatni parametr wyjścia!
    \begin{minted}[fontsize=\footnotesize]{Go}
        func getSmth() error {
            // return fmt.Errorf("formatted error %s", "txt")
        }
        value, err := getValue()
        err = getSmth()
        a, b, c, err := getSmth3()
    \end{minted}

    \begin{alertblock}{Uwaga!}
        Nie należy pomijać obsługi błędów (np. poprzez zapis \quotes{a, \_ := ...})!
    \end{alertblock}

    \begin{alertblock}{Uwaga 2!}
        Nie rzucamy \quotes{panic} jeżeli nie musimy! Lepiej zwrócić błąd.
    \end{alertblock}

    \begin{block}{Tekst errorów}
        Błędy powinny zaczynać się z malej litery i nie kończymy go kropką:
        \quotes{this is correct error message}.
    \end{block}
\end{frame}

\begin{frame}[fragile]
    \frametitle{Defer, panic oraz recover}
    Istnieją sytuacje w kodzie, gdzie musimy coś zrobić za wszelką cenę. Takim przykładem
    jest otwarcie pliku. Jeżeli uda nam się utworzyć plik, to musimy go zamknąć.

    \begin{minted}[fontsize=\footnotesize]{Go}
    src, err := os.Open(srcName)
    if err != nil {
        return
    }
    // Long function
    src.Close() // Wrong, function can be closed before!
    \end{minted}

    Ułatwieniem jest \quotes{defer}, który przeniesie wywołanie funkcji na sam koniec
    jej wywołania (w tym panic także się łapie):

    \begin{minted}[fontsize=\footnotesize]{Go}
        src, err := os.Open(srcName)
        if err != nil {
            return
        }
        defer src.Close() // No mather what, it'll be called
        // Long function
    \end{minted}
\end{frame}

\begin{frame}[fragile]
    \frametitle{Defer, panic oraz recover II}
    Czasami zdarza się tak, że jesteśmy w sytuacji bez wyjścia, gdzie my jako osoby piszące
    funkcję oraz osoby osoby, które ją wywołały nie możemy obsłużyć błędu. Wtedy
    właśnie możemy spanikować. 

    \begin{minted}[fontsize=\footnotesize]{Go}
        panic("message")
    \end{minted}

    Panic jest pewnego rodzaju wyjątkiem, który wystąpił w aplikacji.
    Ponieważ jego obsługa nie jest prosta, należy wywoływać go w \quotes{wyjątkowych}
    sytuacjach.

    \begin{block}{Używaj errora zamiast panikować}
        Programista powinien stosować errory, zamiast przerywać bieg programu.
    \end{block}
\end{frame}

\begin{frame}[fragile]
    \frametitle{Defer, panic oraz recover III}
    Recover ułatwia nam odzyskanie pracy programu po wystąpieniu paniki.
    Jeżeli program panikuje i trafi na recover, jego przepływ wróci do normy.

    Recover może być wywołany tylko w funkcji \quotes{defer}.

    \begin{minted}[fontsize=\footnotesize]{Go}
        func test() {
            defer func() {
                if r := recover(); r != nil {
                    fmt.Println("Recovered in f", r)
                }
            }()
            panic("test")
        }
    \end{minted}

    \begin{block}{Recovery + panic a wyjątki}
        Zauważmy, że nie wiemy dokładnie kto i gdzie tutaj spanikował, przez co trudno obsługiwać
        i odzyskiwać \quotes{flow} takiego programu.
    \end{block}
\end{frame}

\begin{frame}[fragile]
    \frametitle{Importowanie}
    Importowanie paczek zaczyna się słowem kluczowym \quotes{import}. Przy importowaniu
    ostatni człon importowanej paczki tworzy globalną zmienną:

    \begin{minted}[fontsize=\footnotesize]{Go}
        import "github.com/sonyarouje/simdb/db" // Makes db variable
    \end{minted}

    Oczywiście, mogą zdarzyć się konflikty, wtedy należy zmienić nazwę importu:
    \begin{minted}[fontsize=\footnotesize]{Go}
        import newName "github.com/sonyarouje/simdb/db" 
        // Makes newName variable
    \end{minted}

    Importy można łączyć:
    \begin{minted}[fontsize=\footnotesize]{Go}
        import (
            "a"
            "b"
        )
    \end{minted}

    Zewnętrzne paczki instaluje się za pomocą polecenie \quotes{go get}:
    \begin{minted}[fontsize=\footnotesize]{shell}
        go get github.com/mattbaird/jsonpatch
    \end{minted}
\end{frame}

\begin{frame}[fragile]
    \frametitle{Importowanie II}
    Dodatkowo istnieje zapis, kóry powoduje, że wszystkie wyeksportowanie (pisanej z wielkiej litery) funkcje oraz
    zmienne są dostępne bez prefiksu paczki:

    \begin{minted}[fontsize=\footnotesize]{Go}
        import . "b"
    \end{minted}

    \begin{alertblock}{Uwaga!}
        Używanie tego poza testami jest uznawane jako anty-pattern. Czasami używa się
        tego zapisu, aby ułatwić pisanie testów, które trafiły do innej paczki z powodu
        circular dependency.       
    \end{alertblock}

    \begin{alertblock}{Uwaga 2!}
        Paczki powinny być dobrze nazwane, zmiana nazw paczek bez konfliktów powinno 
        występować bardzo rzadko.
    \end{alertblock}

    \begin{alertblock}{Uwaga 3!}
        Jeżeli eksportujemy coś w paczce, nie duplikujemy nazewnictwa. Przykładowo, w
        paczce \quotes{Math} nie tworzymy typów \quotes{MathNumber}. Takie coś zmniejsza
        czytelność kodu: \quotes{math.MathNumber} vs \quotes{math.Number}.
    \end{alertblock}
\end{frame}

\begin{frame}[fragile]
    \frametitle{Importowanie II}
    Widać po powyższych przykładach, że nie ma wersjonowania bibliotek. Go potrafi wersjonować biblioteki,
    lecz robi to inaczej niż reszta języków. Według GoLanga, dopóki ścieżka importu
    jest taka sama, nowa wersja paczki musi być wstecz kompatybilna.

    Jeżeli chcemy wydać nową wersję ze złamaniem kompatybilności, zmieniamy ścieżkę 
    importu:

    \begin{minted}[fontsize=\footnotesize]{shell}
        github.com/googleapis/gax-go @ master branch
        /go.mod    → module github.com/googleapis/gax-go
        /v2/go.mod → module github.com/googleapis/gax-go/v2
    \end{minted}

\end{frame}
\end{document}
